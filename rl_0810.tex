\documentclass[10pt]{beamer}
\usepackage{kotex}
\usepackage{minted}
\title{RL with initial quess}
\author{Yoon-gu Hwang}
\institute{LG CNS}
\date{\today}

\begin{document}
\begin{frame}\frametitle{Docker}
주로 Ubuntu를 사용한다.
\begin{itemize}
	\item On different OS, same environment
	\item Isolated from the host OS
	\item Can be run on a cloud
\end{itemize}

다음 2가지 용어를 구분하자. 간단하게 다음과 같이 비유할 수 있다.

\begin{enumerate}
	\item Docker Image: Windows Backup 이미지
	\item Docker Container: Windows Backup 이미지로 초기화한 컴퓨터
\end{enumerate}
\end{frame}

\begin{frame}[fragile]\frametitle{Dockerfile}

\begin{minted}
[
frame=lines,
framesep=2mm,
baselinestretch=1.2,
fontsize=\footnotesize,
linenos
]
{docker}
FROM pytorch/pytorch:2.0.1-cuda11.7-cudnn8-runtime
RUN apt-get update
RUN apt install -y git
RUN git clone https://github.com/DLR-RM/stable-baselines3.git
RUN cd stable-baselines3 && pip install '.[extra_no_roms]'
RUN pip install hydra-core scipy seaborn
WORKDIR /workspace

\end{minted}

\end{frame}

\begin{frame}[fragile]\frametitle{Create Docker Image}
\begin{enumerate}
	\item \mintinline{bash}{git clone https://github.com/yoon-gu/ezlab-rl}
	\item \mintinline{bash}{cd ezlab-rl}
	\item \mintinline{bash}{docker build . -t sb3}
	\item You can check your created docker image named sb3 by running \mintinline{bash}{docker images}.
\end{enumerate}

\end{frame}

\begin{frame}[fragile]\frametitle{Create Docker Container and Run your code}
\begin{minted}
[
frame=lines,
framesep=2mm,
baselinestretch=1.2,
fontsize=\footnotesize,
linenos
]
{bash}
docker run --gpus all -it \
	-v /home/yoon-gu/ezlab-rl:/workspace \
	sb3 bash
python sir_ppo.py
\end{minted}


\end{frame}
\end{document}